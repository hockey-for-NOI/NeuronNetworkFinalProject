\documentclass{article}

% if you need to pass options to natbib, use, e.g.:
% \PassOptionsToPackage{numbers, compress}{natbib}
% before loading nips_2016
%
% to avoid loading the natbib package, add option nonatbib:
% \usepackage[nonatbib]{nips_2016}

%\usepackage{nips_2016}

% to compile a camera-ready version, add the [final] option, e.g.:
\usepackage[final]{Project_Proposal}

\usepackage[utf8]{inputenc} % allow utf-8 input
\usepackage[T1]{fontenc}    % use 8-bit T1 fonts
\usepackage{hyperref}       % hyperlinks
\usepackage{url}            % simple URL typesetting
\usepackage{booktabs}       % professional-quality tables
\usepackage{amsfonts}       % blackboard math symbols
\usepackage{nicefrac}       % compact symbols for 1/2, etc.
\usepackage{microtype}      % microtypography
\usepackage{CJKutf8}

\begin{CJK*}{UTF8}{gbsn}
\title{《人工神经网络》大作业开题报告}


% The \author macro works with any number of authors. There are two
% commands used to separate the names and addresses of multiple
% authors: \And and \AND.
%
% Using \And between authors leaves it to LaTeX to determine where to
% break the lines. Using \AND forces a line break at that point. So,
% if LaTeX puts 3 of 4 authors names on the first line, and the last
% on the second line, try using \AND instead of \And before the third
% author name.

\author{
  刘熙航 \\
  计算机科学与技术系 \\
  清华大学 \\
  \texttt{liuxh15@mails.tsinghua.edu.cn} \\
  %% examples of more authors
  \AND
  何琦\\
  计算机科学与技术系 \\
  清华大学 \\
  \texttt{hq15@mails.tsinghua.edu.cn} \\
  %% \And
  %% Coauthor \\
  %% Affiliation \\
  %% Address \\
  %% \texttt{email} \\
}

\begin{document}

\maketitle



\section{任务定义}
\subsection{任务背景}

与传统任务已有确定数据集的情况不同,在真实的场景里,数据是不断产生的,而数据的质量也会随着时间不断变化,如下面两种场合:

\begin{itemize}
    \item 在自动驾驶的数据标注中,需要判断人被车辆遮挡的情况,在一开始,数据标注所采用的label为“标注员所认为的人全身的位置”。然而在一段时间之后,研究员发现,对于该任务而言,标注“人没有被车遮挡住的部分”的效果会远好于之前的label,然而数据的重新标注所需要的时间是很长的,在这样的情况下,我们应当如何利用旧标签与新标签的强相关性去尽快训练一个满足新标签的模型?
    \item 对一个推荐系统而言,用户的喜好是不断变化的,换言之,新数据相对旧数据而言更有价值。但是如果仅仅使用新数据,可能会由于样本量过小,以致于难以通过模型得到合适的推荐信息。在这种情况下,如何利用数据不断变化,但变化程度随时间幅度较小的特性,去推断用户所感兴趣的内容?
\end{itemize}

\subsection{任务描述}
首先,会按照数据与最终数据的相似程度,将其分为n个stage。之后,将会完成以下两个任务。

\textbf{任务一:}

\textbf{在n个stage的数据上训练n-2个神经网络,并将第i个stage的数据训练所得到的网络在相邻的stage(即stage \#(i-1)和stage \#(i+1))上进行测试。}

\textbf{任务二:}

该任务首先会训练两个模型,

A模型:在预训练阶段,可以无限获取前n-1个stage的全部数据,在正式训练时可以获取n个stage的全部数据。

B模型:不经过预训练,正式训练时可以获取第n个stage的数据。

该任务的内容为:\textbf{得出一个策略$\alpha*$,使得该策略下,预训练之后的A模型,可以通过相同的迭代次数,得到超过B模型的表现。}

我们的任务将采用准确度较高,可以定量施加影响的数据集。在此基础上,对数据施加定量的影响,从而进行研究。
该任务将按照Classification和Regression分成两个任务进行。具体施加影响的方式参见第二部分“数据集”。

\section{数据集}

针对该任务,我们会将其分成Classification和Regression两部分

\subsection{Regression}
这部分数据集我们将采用定长定随机序列。

Initial Label:数据的40\%分位数。

Final Label:数据的60\%分位数。

之后,数据将被分为n个stage,同样的,设数据大小为m,则每个stage的数据规模为$\frac{m}{n}$。每个stage对应的label如下:

Stage \#0: $40$\%分位数。

Stage \#1: $40+20\times \frac{1}{n}$分位数。

Stage \#2: $40+20 \times \frac{2}{n}$分位数。

Stage \#3: $40+20 \times \frac{3}{n}$分位数。

...

Stage \#(n-1): $40+20 \times \frac{n-1}{n}$分位数。

Test Set: Final Label.

\subsection{Classification}
这部分我们将采用经典的MNIST数据集,并对数据集进行如下的划分:

首先,对label进行如下的定义:

Initial Label:初始label

Final Label:初始label xor 1之后得到的结果。

之后,我们会将训练数据分为n个stage,每个stage均包含有$\frac{60000}{n+1}$张图片,之后对label进行以下的改动(数据不做改变):

Stage \#0: $\frac{n}{n}$ Initial Label, $\frac{0}{n}$ Final Label.

Stage \#1: $\frac{n-1}{n}$ Initial Label, $\frac{1}{n}$ Final Label.

Stage \#2: $\frac{n-2}{n}$ Initial Label, $\frac{2}{n}$ Final Label.

...

Stage \#(n-2): $\frac{2}{n}$ Initial Label, $\frac{n-2}{n}$ Final Label.

Stage \#(n-1): $\frac{1}{n}$ Initial Label, $\frac{n-1}{n}$ Final Label.

Stage \#n: $\frac{0}{n}$ Initial Label, $\frac{n}{n}$ Final Label.

Test Set: $\frac{n}{n}$ Final Label.

其中A模型可以预先在Stage \# 0 至 Stage \#(n)上进行训练,而B模型仅能在Stage \#n进行训练

\section{挑战和基线}

\subsection{挑战}

我们认为,挑战主要有以下几个方面:

\begin{itemize}
    \item 我们能否找到能够优化表现的训练策略?
    \item 不同的模型是否会需要不同的优化策略,如果是的话,寻找多个策略所需要的时间成本是否可以接受?
    \item 在真实的需求里,数据的质量是无法被定量刻画的,那么我们的策略能否提升模型在难以定量刻画影响的真实数据上的表现?
\end{itemize}

\subsection{基线}

我们认为,对于任务一而言,其重点在于研究label的错误标注的影响,故该任务不需要相应的Baseline。
但对于任务二而言,其最基础的目标是,通过与n个不同stage数据集的交互,可以得到一个比仅通过第n个数据集训练所得到的模型有更好的表现。

\section{研究计划}


清楚地描述你的想法,并将其和数学语言甚至代码联系起来。可以使用脚注、图和表来描述你的想法。

\subsection{时间安排}

第九周:完成Project Proposal

第十周:完成数据集的生成和数据集参数的选取。

第十一周:部分完成任务描述中的任务一,并对数据集进行可能的调整,使之可以用于完成后续的任务。并在此外提交项目所要求的Milestone Report。

第十二周:完成任务描述中的任务一。

第十三至十四周:完成任务描述中的任务二。

第十五至十六周:我们的结果在真实的数据集中进行测试。

\section{可行性}

我们认为,该任务的可行总体而言较好,但是该任务中对“策略”的定义较为模糊。此外,两周的时间可能较难以完成从数据选取到模型训练的整个流程的优化,这也是导致项目失败的可能性最大的原因。

此外,由于该项目的目的在于解决一个较为创新的问题,这也意味着,其无法保证会在预期的时间里产生成果。这也可能导致整个项目的失败。

\section*{参考文献}


\medskip

\small

[1] Alexander, J.A.\ \& Mozer, M.C.\ (1995) Template-based algorithms
for connectionist rule extraction. In G.\ Tesauro, D.S.\ Touretzky and
T.K.\ Leen (eds.), {\it Advances in Neural Information Processing
  Systems 7}, pp.\ 609--616. Cambridge, MA: MIT Press.

[2] Bower, J.M.\ \& Beeman, D.\ (1995) {\it The Book of GENESIS:
  Exploring Realistic Neural Models with the GEneral NEural SImulation
  System.}  New York: TELOS/Springer--Verlag.

[3] Hasselmo, M.E., Schnell, E.\ \& Barkai, E.\ (1995) Dynamics of
learning and recall at excitatory recurrent synapses and cholinergic
modulation in rat hippocampal region CA3. {\it Journal of
  Neuroscience} {\bf 15}(7):5249-5262.

\end{document}

\end{CJK*}